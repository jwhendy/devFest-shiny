% Created 2014-02-07 Fri 23:00
\documentclass[sans,aspectratio=169,presentation,bigger,fleqn]{beamer}
\usepackage[utf8]{inputenc}
\usepackage[T1]{fontenc}
\usepackage{fixltx2e}
\usepackage{graphicx}
\usepackage{longtable}
\usepackage{float}
\usepackage{wrapfig}
\usepackage{rotating}
\usepackage[normalem]{ulem}
\usepackage{amsmath}
\usepackage{textcomp}
\usepackage{marvosym}
\usepackage{wasysym}
\usepackage{amssymb}
\usepackage{hyperref}
\tolerance=1000
\usepackage{setspace}
\setstretch{1.3}
\usepackage{booktabs}
\hypersetup{colorlinks=true,linkcolor=blue,urlcolor=blue}
%\usetheme{naked}
\usepackage{lmodern}
\usetheme[alternativetitlepage=true,titleline=true]{Torino}
\usecolortheme{freewilly}
\usetheme{default}
\author{John Henderson}
\date{08 February 2014}
\title{An introduction to Shiny}
\hypersetup{
  pdfkeywords={},
  pdfsubject={},
  pdfcreator={Emacs 24.3.1 (Org mode 8.2.5h)}}
\begin{document}

\maketitle

\begin{frame}[fragile,label=sec-1]{Intro}
 \begin{itemize}
\item \texttt{shiny} is an \texttt{R} package that enables web based applications
\item Overview of \texttt{shiny} basics
\item Three examples
\item The code/data necessary to reproduce anything in this talk is on \href{https://github.com/jwhendy/devFest-shiny}{github}
\end{itemize}
\end{frame}
\begin{frame}[fragile,label=sec-2]{Basics}
 \begin{itemize}
\item \texttt{shiny} works inside of \href{http://www.rstudio.com/}{RStudio}

\item Two files are required to run an application
\item \texttt{ui.R}: page format, user inputs, and outputs you're going to create
\item \texttt{server.R}: contains the R code which will generate your dynamic output
\end{itemize}

\vspace{0.5cm}

Don't forget to run \texttt{install.packages("shiny")}!
\end{frame}
\begin{frame}[fragile,label=sec-3]{Minimal \texttt{ui.R}}
 \scriptsize
\begin{verbatim}
library(shiny)
# page format
shinyUI(pageWithSidebar(
  # title
  headerPanel("Hello Shiny!"),

  sidebarPanel(
    # user inputs go here
  ),

  mainPanel(
    plotOutput("plot") # what you're going to output, e.g. a plot
  )
))
\end{verbatim}
\scriptsize
\end{frame}
\begin{frame}[fragile,label=sec-4]{Minimal \texttt{server.R}}
 \scriptsize
\begin{verbatim}
library(shiny)

shinyServer(function(input, output) {

  # general R code here: load libraries, set variables/functions/etc.

  # output$name has to match ui.R's plotOutput("name")
  output$plot <- renderPlot({

    # code to make a plot goes here

  })
})
\end{verbatim}
\scriptsize
\end{frame}
\begin{frame}[fragile,label=sec-5]{It works!}
 \begin{itemize}
\item After defining the above files\ldots{}
\end{itemize}

\scriptsize
\begin{verbatim}
# run from within R studio
library(shiny)
setwd("/path/to/ui-and-server.R")
runApp()
\end{verbatim}

\begin{center}
\includegraphics[height=3.75cm]{./img/shiny-template.png}
\end{center}
\end{frame}
\begin{frame}[label=sec-6]{Example: data analysis/exploration}
\begin{itemize}
\item Enable rapid and dynamic switching of plot variables
\item Allows for "plot prototyping" to examine trends/relationships
\item Web-based solution is easily sharable with others
\end{itemize}
\end{frame}
\begin{frame}[fragile,label=sec-7]{Fiddling with public transportation data}
 \begin{itemize}
\item Grabbed data on public transportation centers around US (more \href{https://github.com/tcrug/public-transpo}{here})
\item Some are quite efficient, some are horrible
\item Can \texttt{shiny} help find some interesting tidbits?
\end{itemize}

\pause

\href{http://jwhendy.shinyapps.io/transpo-exploration/}{\alert{Demo time!}}
\end{frame}
\begin{frame}[fragile,label=sec-8]{Example: interactive contour plots}
 \begin{itemize}
\item Applied machine learning in \texttt{R} on product test data
\item Contour plots can be nice for visualizing effect of inputs vs. outputs
\item How to share the results with co-workers who don't use \texttt{R}?
\end{itemize}

\pause

\href{http://spark.rstudio.com/jwhendy/interactive-contour/}{\alert{Demo time!}}
\end{frame}
\begin{frame}[label=sec-9]{Example: visualizing insurance costs}
\begin{itemize}
\item Benefit plan choices are tough!
\item Started making visualizations/walkthroughs at 3M in 2011
\item Goal: simplify decision process through visualization
\end{itemize}
\end{frame}
\begin{frame}[label=sec-10]{The main issue}
\begin{itemize}
\item HR typically sends you a table like this on glossy paper; which plan is best?
\end{itemize}

\begin{center}
\begin{tabular}{lll}
\toprule
 & Plan A & Plan B\\
\midrule
Premium & \$150/mo & \$250/mo\\
3M Contribution & \$1,000 & \$0\\
Deductible & \$2,500 & \$750\\
\(OOP_{max}\) & \$5,000 & \$4,000\\
\bottomrule
\end{tabular}
\end{center}
\end{frame}
\begin{frame}[label=sec-11]{The main issue}
\begin{itemize}
\item These employees are not smiling because they understood the table
\end{itemize}

\begin{center}
\includegraphics[height=5cm]{./img/choosing-insurance.jpg}
\end{center}

\tiny
Image credit: \url{http://jtsfs.com/employee-benefits-2/group-health-insurance/}
\end{frame}
\begin{frame}[label=sec-12]{In 2011, it was so simple!}
\begin{center}
\includegraphics[height=6.5cm]{./img/ins-intersections.pdf}
\end{center}
\end{frame}
\begin{frame}[label=sec-13]{Fast-foward to 2013}
\begin{itemize}
\item 3M introduces split deductibles on two plans
\item Now which plan is best?
\end{itemize}

\footnotesize
\begin{center}
\begin{tabular}{lllllll}
\toprule
Plan & Premium & \(Ded_{ind}\) & \(Ded_{tot}\) & \(OOP_{ind}\) & \(OOP_{tot}\) & \(HSA\)\\
\midrule
A & \$3,500 & \$500 & \$1,000 & \$2,000 & \$4,000 & -\\
B & \$2,200 & - & \$2,750 & - & \$5,500 & \$1,250\\
C & \$600 & \$2,750 & \$5,500 & \$5,500 & \$11,000 & \$1,250\\
\bottomrule
\end{tabular}
\end{center}
\normalsize
\end{frame}
\begin{frame}[label=sec-14]{First shot}
\begin{itemize}
\item Now we need axes for max spender vs. everyone else\ldots{} contour plot!
\end{itemize}

\begin{center}
\includegraphics[height=5.5cm]{./img/ins-contour.pdf}
\end{center}
\end{frame}
\begin{frame}[label=sec-15]{Winning cost map}
\begin{itemize}
\item "Stack" the contours, figure out which one is lowest
\end{itemize}

\begin{center}
\includegraphics[height=5.5cm]{./img/ins-cost-map.pdf}
\end{center}
\end{frame}
\begin{frame}[fragile,label=sec-16]{So, what about \emph{this} year?}
 \begin{itemize}
\item I used \texttt{shiny}, obviously!
\item Dynamic UI elements for \# of people on plan
\item \href{http://stackoverflow.com/questions/18116967/dealing-with-conditionals-in-a-better-manner-than-deeply-nested-ifelse-blocks}{"Interesting" algorithm} for dealing with complex criteria
\item Hosted internally at 3M  with \texttt{shiny-server}
\item Put an anonymized version on \href{http://spark.rstudio.com/jwhendy/insurance-visualizer}{RStudio server}
\end{itemize}
\end{frame}
\begin{frame}[label=sec-17]{Table of possible outcomes}
\begin{center}
\tiny
\begin{center}
\begin{tabular}{rrrrrrrl}
\toprule
ded\(_{\text{ind}}\) & oop\(_{\text{ind}}\) & ded\(_{\text{rem}}\) & oop\(_{\text{rem}}\) & ded\(_{\text{tot}}\) & oop\(_{\text{tot}}\) & bin & formula\\
\midrule
0 & 0 & 0 & 0 & 0 & 0 & 0 & exp\(_{\text{ind}}\) + exp\(_{\text{rem}}\)\\
1 & 0 & 0 & 0 & 0 & 0 & 1 & ded\(_{\text{ind}}\) + 0.1 (exp\(_{\text{ind}}\) - ded\(_{\text{ind}}\)) + exp\(_{\text{rem}}\)\\
0 & 0 & 1 & 0 & 0 & 0 & 4 & exp\(_{\text{ind}}\) + exp\(_{\text{rem}}\)\\
1 & 0 & 0 & 0 & 1 & 0 & 17 & ded\(_{\text{ind}}\) + 0.1 (exp\(_{\text{ind}}\) - ded\(_{\text{ind}}\)) + exp\(_{\text{rem}}\)\\
1 & 1 & 0 & 0 & 1 & 0 & 19 & oop\(_{\text{ind}}\) + exp\(_{\text{rem}}\)\\
0 & 0 & 1 & 0 & 1 & 0 & 20 & ded\(_{\text{tot}}\) + 0.1 (exp\(_{\text{ind}}\) + exp\(_{\text{rem}}\) - ded\(_{\text{tot}}\))\\
1 & 0 & 1 & 0 & 1 & 0 & 21 & ded\(_{\text{tot}}\) + 0.1 (exp\(_{\text{ind}}\) + exp\(_{\text{rem}}\) - ded\(_{\text{tot}}\))\\
1 & 1 & 1 & 0 & 1 & 0 & 23 & oop\(_{\text{ind}}\) + ded\(_{\text{ind}}\) + 0.1 (exp\(_{\text{rem}}\) - ded\(_{\text{ind}}\))\\
1 & 0 & 1 & 1 & 1 & 0 & 29 & ded\(_{\text{tot}}\) + 0.1 (exp\(_{\text{ind}}\) + exp\(_{\text{rem}}\) - ded\(_{\text{tot}}\))\\
1 & 1 & 0 & 0 & 1 & 1 & 51 & oop\(_{\text{ind}}\) + exp\(_{\text{rem}}\)\\
1 & 1 & 1 & 0 & 1 & 1 & 55 & oop\(_{\text{ind}}\) + ded\(_{\text{ind}}\) + 0.1 (exp\(_{\text{rem}}\) - ded\(_{\text{ind}}\))\\
1 & 0 & 1 & 1 & 1 & 1 & 61 & oop\(_{\text{tot}}\)\\
1 & 1 & 1 & 1 & 1 & 1 & 63 & oop\(_{\text{tot}}\)\\
\bottomrule
\end{tabular}
\end{center}
\normalsize
\end{center}
\end{frame}
\begin{frame}[fragile,label=sec-18]{Check against criteria; convert to binary}
 \scriptsize
\begin{verbatim}
  test_case <- c(rep(c(exp_ind, exp_rem, exp_ind + exp_rem),   # vector of predicted costs
                     each = 2))                                # for max vs. others

  test_case <- rbind(test_case, test_case, test_case)          # three sets for three plans
  
  limits <- cbind(compare$ded_ind, compare$exp_max_ind,        # criteria values
                  compare$ded_ind, compare$exp_max_ind, 
                  compare$ded_tot, compare$exp_max_tot)
  
  result <- cbind(compare[, c("ded_ind", "ded_tot", "oop_ind", # store cutoffs in result
                              "oop_tot", "prem", "hsa")],
                  exp_ind, exp_rem,
                  (test_case > limits) %*% (2^(0:5)))          # convert T/F to binary
\end{verbatim}
\normalsize
\end{frame}
\begin{frame}[fragile,label=sec-19]{Hacky function lookup}
 \tiny
\begin{verbatim}
map_funcs <- list(
  "0" = function(binary) { binary$exp_ind + binary$exp_rem }, 
  "1" = function(binary) { binary$ded_ind + (0.1* (binary$exp_ind - binary$ded_ind)) + binary$exp_rem }, 
  "4" = function(binary) { binary$exp_ind + binary$exp_rem }, 
  "16" = function(binary) { binary$ded_tot + (0.1 * (binary$exp_ind + binary$exp_rem - binary$ded_tot)) },
  "17" = function(binary) { binary$ded_ind + (0.1* (binary$exp_ind - binary$ded_ind)) + binary$exp_rem },
  "19" = function(binary) { binary$oop_ind + binary$exp_rem }, 
  "20" = function(binary) { binary$ded_tot + (0.1 * (binary$exp_ind + binary$exp_rem - binary$ded_tot)) }, 
  "21" = function(binary) { binary$ded_tot + (0.1 * (binary$exp_ind + binary$exp_rem - binary$ded_tot)) }, 
  "23" = function(binary) { binary$oop_ind + binary$ded_ind + (0.1 * (binary$exp_rem - binary$ded_ind)) },
  "28" = function(binary) { binary$ded_tot + (0.1 * (binary$exp_ind + binary$exp_rem - binary$ded_tot)) },
  "29" = function(binary) { binary$ded_tot + (0.1 * (binary$exp_ind + binary$exp_rem - binary$ded_tot)) },
  "48" = function(binary) { binary$oop_tot },   
  "51" = function(binary) { binary$oop_ind + binary$exp_rem }, 
  "55" = function(binary) { binary$oop_ind + binary$ded_ind + (0.1 * (binary$exp_rem - binary$ded_ind)) }, 
  "60" = function(binary) { binary$oop_tot }, 
  "61" = function(binary) { binary$oop_tot }, 
  "63" = function(binary) { binary$oop_tot }
)
\end{verbatim}
\normalsize
\end{frame}
\begin{frame}[label=sec-20]{}
\vfill
\begin{center}
'Nuff talk, let's take a \href{http://spark.rstudio.com/jwhendy/insurance-visualizer}{look}!
\end{center}
\vfill
\end{frame}
\begin{frame}[fragile,label=sec-21]{Sharing \texttt{shiny} apps}
 \begin{itemize}
\item Method 1: tar/zip all files, send, have user run locally
\item Method 2: install \href{http://www.rstudio.com/shiny/server/}{shiny-server} on local machine

\item Method 3: request account for RStudio server account (still available?)
\begin{itemize}
\item Create/upload files; \url{http://spark.rstudio.com/uname/appName}
\end{itemize}

\item Method 4: request account on \emph{new} RStudio server \href{http://www.shinyapps.io/signup.html}{here}
\begin{itemize}
\item Create apps locally, then follow \href{https://github.com/rstudio/shinyapps/}{shinyapps} instructions
\item When satisfied, just run \texttt{deployApp()}!
\item Visit app at \url{http://uname.shinyapps.io/appName/}
\end{itemize}
\end{itemize}
\end{frame}
\begin{frame}[fragile,label=sec-22]{References}
 \begin{itemize}
\item \href{http://www.rstudio.com/shiny/}{Getting started} with \texttt{shiny}
\item \texttt{shiny} \href{https://groups.google.com/forum/#!forum/shiny-discuss}{mailing list}
\item RStudio server \href{https://shinyapps.io/}{application}
\item \href{http://stackoverflow.com/questions/19130455/create-dynamic-number-of-input-elements-with-r-shiny}{SO question} on creating dymanic input elements
\item \href{http://stackoverflow.com/questions/17683933/set-global-object-in-shiny}{SO question} on global variables (not intuitive!)
\item \href{http://stackoverflow.com/questions/17838709/scale-and-size-of-plot-in-rstudio-shiny}{SO question} on sizing plots in \texttt{shiny}
\item \href{http://stackoverflow.com/questions/17958730/faceting-a-set-of-contour-plots-in-ggplot-r}{SO question} that solved my contour plot issue; repaid with \texttt{shiny} example
\end{itemize}
\end{frame}
\begin{frame}[label=sec-23]{Apps in this presentation}
\begin{itemize}
\item Transpo exploration: \href{http://spark.rstudio.com/jwhendy/transpo-exploration/}{spark.rstudio} or \href{http://jwhendy.shinyapps.io/transpo-exploration}{shinyapps.io}
\item \href{http://spark.rstudio.com/jwhendy/interactive-contour/}{Interactive contour}
\item \href{http://spark.rstudio.com/jwhendy/insurance-visualizer/}{Benefit analysis}
\item Everything's also on \href{https://github.com/jwhendy/devFest-shiny}{github}!
\end{itemize}
\end{frame}
% Emacs 24.3.1 (Org mode 8.2.5h)
\end{document}
